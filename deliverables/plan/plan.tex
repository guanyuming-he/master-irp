\documentclass[project-plan]{report-template}
% \documentclass[final-report]{report-template}

\usepackage{graphicx}
\usepackage{amsmath}

\graphicspath{{./figures/}}

\university{Imperial College London}
\department{Department of Earth Science and Engineering}
\course{MSc in Applied Computational Science and Engineering}
\title{Current Content Discovery for Module Teaching}
\author{Guanyuming He}
\email{guanyuming.he24@imperial.ac.uk}
\githubusername{esemsc-gh124}
\supervisors{Sean O'Grady\\
             Rhodri Nelson}
\repository{https://github.com/ese-ada-lovelace-2024/irp-gh124}

\newcommand\casemethod{case method}

\begin{document}

\maketitlepage  

\section*{Abstract}
TBD.
\textbf{Keywords:} content discovery, information retrieval, LLM, search engines, \casemethod, Business school,

\section{Introduction}
Since the emergence of the first Business schools in the late 19th century
\cite{first.bis.school.1, first.bis.school.2}, several distinct pedagogical
teaching strategies have been applied.  First institutionalized at Harvard
Business School \cite{case.method.origin.1, case.method.origin.2} in the early
20th century, a method about teaching students with real world business cases
(will be called \emph{\casemethod} in the rest of the thesis) has been found
more effective and engaging \cite{case.method.support.1, case.method.support.2,
case.method.support.3} than many traditional, for example, big lecture based,
teaching methods. The \casemethod\ is valued as a form of \emph{active
learning} methods for students, for its ability to expose students to complex,
context-specific problems that lack clear-cut solutions. As such, it has found
wide adoption across the world \cite{case.method.adoption.1,
case.method.adoption.2}.

However, despite its aforementioned adoption and performance in business school
teaching, the case method faces a significant constraint: the availability and
collection of timely and relevant case material \cite{case.method.limit.1,
case.method.limit.3}. In particular, Christensen has identified in his classical
article that instructors would have to conduct ``extensive preparation''
\cite{case.method.limit.2} for case method. Another factor contributing to this
constraint of case method is the ever-evolving business world and the necessity
of the latest information: Clark argues that, because learned skill will lose
value quickly in five years, it is critical for students to be up to date to
remain relevant in the business world \cite{case.method.limit.4}; McFarlane
emphasises the importance of updated cases, as otherwise students could be
disengaged or discouraged \cite{case.method.limit.5}.

During the past two centuries, a number of key developments have profoundly
expanded an individual's capacity to retrieve information about the world. In
the early 19th century, the transmission of information was constrained by the
physical limitations of transportation and postal delivery; messages had to be
carried on paper or remembered by a person, which would often need days or
weeks to reach their recipients. This fundamentally limited both the speed and
geographic reach of information retrieval. The invention of telegraph in the
1840s by Morse, Cornell, and Henry \cite{history.telegraph.1,
history.telegraph.2}, notably with Samuel Morse's first telegraph message,
``\emph{What hath God wrought?}'' in 1844 \cite{first.telegraph.msg}, marked a
paradigm shift by enabling nearly instantaneous transmission of textual
information over relatively short distances (w.r.t.\ the earth) via wired
networks.  This technological innovation was considerably improved by the
invention of the telephone by Alexander Graham Bell in 1876
\cite{history.telephone.1, history.telephone.2}, enabling communication
directly by human voice, instead of encoded Morse code. 

Although wired communication was a giant leap in speed
of information spread, it was still critically constrianed by geographical
features on where the wires were laid. Around the late 19th century, Guglielmo
Marconi's experiments with wireless telegraphy \cite{history.wireless.1} and
the first successful transatlantic signal in 1901
\cite{history.first.atlantic.broadcast} introduced electromagnetic wave-based
wireless communication, eventually accumulating into the world's first voice
broadcast by radio in 1906 \cite{first.voice.broadcast}. These milestones
collectively redefined the temporal and spatial boundaries of information
access, transforming information retrieval from a manual and delayed process
into a near-instantaneous activity acrossing even continents.

The next many decades have seen people improving on the then limitations of
these methods. In particular, the carrying capacity of the wireless waves.

\section{Methods}
Abc

\section{Results}

\section{Discussion}

\section{Conclusion}

\clearpage

% References
\bibliographystyle{plain}
\bibliography{references}  % BibTeX references are saved in references.bib

\end{document}          
