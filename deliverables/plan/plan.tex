\documentclass[project-plan]{report-template}
% \documentclass[final-report]{report-template}

\usepackage{graphicx}
\usepackage{amsmath}

\graphicspath{{./figures/}}

\university{Imperial College London}
\department{Department of Earth Science and Engineering}
\course{MSc in Applied Computational Science and Engineering}
\title{Current Content Discovery for Module Teaching}
\author{Guanyuming He}
\email{guanyuming.he24@imperial.ac.uk}
\githubusername{esemsc-gh124}
\supervisors{Sean O'Grady\\
             Rhodri Nelson}
\repository{https://github.com/ese-ada-lovelace-2024/irp-gh124}

\begin{document}

\maketitlepage  

\section*{Abstract}
TBD.

\section{Introduction}
Since the emergence of the first Business schools in the late 19th century
\cite{first.bis.school.1, first.bis.school.2},
several distinct pedagogical teaching strategies have been applied. Among these
strategies, the recent case study method, which is about using real world business
cases to teach students, has found widespread adoption \cite{???}. 
One method that has seen a large adoption \cite{???} in the recent years is the
case study method.  First institutionalized at Harvard Business School in the
early 20th century, this method is designed to simulate managerial
decision-making by presenting students with real or realistic business
situations, encouraging structured analysis and discussion.

The case method is valued for its ability to expose students to complex,
context-specific problems that lack clear-cut solutions \cite{???}. Unlike
lecture-based teaching, which often emphasizes theoretical abstraction
\cite{???}, case-based instruction engages students in applied reasoning and
deliberation within a collaborative classroom setting \cite{???}. Studies in
management education suggest that this form of instruction can support the
development of analytical reasoning, communication skills, and practical
judgment when implemented carefully and with adequate support \cite{???}.

However, despite its continued use and institutional support \cite{???}, the
case method faces a significant constraint: the availability of timely and
relevant case material \cite{???}. Business environments are dynamic, and cases
can quickly lose pedagogical relevance due to changes in market conditions,
regulations, or technology \cite{???}. Additionally, developing new cases
requires substantial faculty time and institutional resources \cite{???}, and
access to proprietary or sensitive company information is often restricted
\cite{???}. These challenges limit the method’s responsiveness to current
developments in business practice and create tension between realism and
feasibility in curricular design \cite{???}.

\section{Methods}
Abc

\section{Results}

\section{Discussion}

\section{Conclusion}

% References
\bibliographystyle{plain}
\bibliography{references}  % BibTeX references are saved in references.bib

\end{document}          
