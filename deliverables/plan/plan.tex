\documentclass[project-plan]{report-template}
% \documentclass[final-report]{report-template}

\usepackage{graphicx}
\usepackage{amsmath}

\graphicspath{{./figures/}}

\university{Imperial College London}
\department{Department of Earth Science and Engineering}
\course{MSc in Applied Computational Science and Engineering}
\title{Current Content Discovery for Module Teaching}
\author{Guanyuming He}
\email{guanyuming.he24@imperial.ac.uk}
\githubusername{esemsc-gh124}
\supervisors{Sean O'Grady\\
             Rhodri Nelson}
\repository{https://github.com/ese-ada-lovelace-2024/irp-gh124}

\newcommand\casemethod{case method}

\begin{document}

\maketitlepage  

\section*{Abstract}
TBD.
\textbf{Keywords:} content discovery, information retrieval, LLM, search engines, \casemethod, Business school,

\section{Introduction}
\subsection{Problem background}
Since the emergence of the first Business schools in the late 19th century
\cite{first.bis.school.1, first.bis.school.2}, several distinct pedagogical
teaching strategies have been applied.  First institutionalized at Harvard
Business School \cite{case.method.origin.1, case.method.origin.2} in the early
20th century, a method about teaching students with real world business cases
(will be called \emph{\casemethod} in the rest of the thesis) has been found
more effective and engaging \cite{case.method.support.1, case.method.support.2,
case.method.support.3} than many traditional, for example, big lecture based,
teaching methods. The \casemethod\ is valued as a form of \emph{active
learning} methods for students, for its ability to expose students to complex,
context-specific problems that lack clear-cut solutions. As such, it has found
wide adoption across the world \cite{case.method.adoption.1,
case.method.adoption.2}.

However, despite its aforementioned adoption and performance in business school
teaching, the case method faces a significant constraint: the availability and
collection of timely and relevant case material \cite{case.method.limit.1,
case.method.limit.3}. In particular, Christensen has identified in his classical
article that instructors would have to conduct ``extensive preparation''
\cite{case.method.limit.2} for case method. Another factor contributing to this
constraint of case method is the ever-evolving business world and the necessity
of the latest information: Clark argues that, because learned skill will lose
value quickly in five years, it is critical for students to be up to date to
remain relevant in the business world \cite{case.method.limit.4}; McFarlane
emphasises the importance of updated cases, as otherwise students could be
disengaged or discouraged \cite{case.method.limit.5}.

\subsection{Past advancements in information retrieval}
During the past two centuries, a number of key developments have profoundly
expanded an individual's capacity to retrieve information about the world. In
the early 19th century, the transmission of information was still traditional
--- carried by person on paper or simply remembered. This fundamentally limited
both the speed and geographic reach of information retrieval. The invention of
telegraph in the 1840s by Morse, Cornell, and Henry \cite{history.telegraph.1,
history.telegraph.2}, notably with Morse's first telegraph message,
``\emph{What hath God wrought?}'' in 1844 \cite{first.telegraph.msg}, marked a
paradigm shift by enabling very fast transmission of encoded information over
relatively short distances (w.r.t.\ the earth) via wired networks.  This
technological innovation was considerably improved by the invention of the
telephone by Bell in 1876 \cite{history.telephone.1, history.telephone.2},
enabling communication directly by human voice, instead of encoded Morse code.

Wired communication is critically constrianed by geographical
features on where the wires were laid. Around the late 19th century, 
Marconi's experiments with wireless telegraphy \cite{history.wireless.1} and
the first successful transatlantic signal in 1901
\cite{history.first.atlantic.broadcast} introduced electromagnetic wave-based
wireless communication, eventually accumulating into the world's first voice
broadcast by radio in 1906 \cite{first.voice.broadcast}. These milestones
collectively redefined the temporal and spatial boundaries of information
access.

The next many decades have seen people improving on the serious limitations of
wireless communication: signal strength, interferences and attenuation,
carrying capacity, and even deliberate sabotage during war times
\cite{wireless.weakness.1, wireless.weakness.2, wireless.weakness.3}.
Theoretically, Hartley observed a logritham pattern of information capacity
\cite{hartley.log.information} and then Shannon expanded on it to first define
\emph{bits} and \emph{entropy}, giving a formal mathematical theory of
information \cite{shannon.theory.communication} in 1948. Meanwhile,
engineers were experimenting with alternative modulation techniques, notably
frequency modulation (FM), and the concepts were formalized in the 1930s
\cite{history.modulation}. 

As the theoretical understanding of information progressed, people began to
have the idea of \emph{searching} for information based on content
and by relevance, instead of by unique identifier
\cite{history.information.retrieval}. Actually, the term \emph{information
retrieval} was not invented until Mooers coined it in 1950
\cite{mooers.info.ret.term}. Since then, information retrieval systems have
quickly evolved, and went through four phases before 2000: ``(1) manual and mechanical
devices; (2) offline computing; (3) online computing, vendor access; (4)
distributed, networked, and mass computing.'' \cite{info.ret.4.phases}, with
the last three substantially contributed to by the invention of the Internet
\cite{history.internet}, and consequently the emergence of search engines in
the 1990s \cite{history.search.engines, history.internet.search.engines}.
For the next two decades, search engines greatly expanded in speed and
coverage and has been significantly impacting the society's information for at
least a decade \cite{search.engine.impact.1, search.engine.impact.2}.

LLMs have undoubtedly entered and transformed many areas, including daily life,
business and the industry, and research \cite{llm.impact.1}. One strong appeal
of LLMs is that they could process user's natural language input and generate
natural language output in return with remarkable resemblence to what a human
would say \cite{llm.power.1, llm.power.2}. However, because of the inherent
limit of neural networks, some argue that they could not formally reason about
what they output \cite{llm.limit.1, llm.limit.2, llm.limit.3}. Indeed, hallucination 
\cite{llm.hallucination.1, llm.hallucination.2} and other forms of distortion
of facts, is a big problem of LLMs. On the other hand, although search engines
rely on determinstic algorithms that give precise reference to searched result,
they could not compare with LLMs' ability to process search prompt and
summarize results. 

Therefore, various attempts have been made to integrate LLMs with search
engines \cite{llm.meet.search.1, llm.meet.search.2, llm.meet.search.3}.
Specifically, Xiong
et al.\ proposes to categorize them into ``LLM4Search'' and ``Search4LLM'',
where ``A4B`` means using A to improve B \cite{llm.meet.search.1}. Here is
where my thesis will build upon. Previously, one would need to carefully craft
search engine prompts into clearly and precise list of words or a short
sentence. Now, with the help with LLMs that can easily process frivolous
natural language input, I plan to integrate them together to boost information
retrieval further, especially in the area of business school teaching content
discovery.

\section{Methods} 

\section{Results}

\section{Discussion}

\section{Conclusion}

\clearpage

% References
\bibliographystyle{plain}
\bibliography{references}  % BibTeX references are saved in references.bib

\end{document}          
